En este trabajo se realiza un estudio de la incertidumbre que presentan juegos
de estrategia en tiempo real como StarCraft, utilizando técnicas estadísticas
y de aprendizaje. Así se consigue evaluar la incertidumbre
que presentan este tipo de problemas para, en un paso posterior de optimización,
poder tenerla en cuenta y poder desarrollar agentes que sean capaces de evaluar
esta incertidumbre y actuar o modificar su conducta según lo necesite.

\begin{otherlanguage}{british}
In this project, it is studied the uncertainty that Real Time Strategy games
like StarCraft present, applying statistical techniques and learning ones. With all of this,
uncertainty can be evaluated for a next optimization, so agents
who can evaluate this uncertainty and modify its strategy could be developed.

\subsubsection*{Statistical Learning}

Firstly, it is defined the concept of \emph{statistical learning}. It is a
branch of Mathematics that tries to learn unknown functions from data using
statistical techniques.

\end{otherlanguage}
