\documentclass[a4paper,11pt]{book}\usepackage[]{graphicx}\usepackage[]{color}
%% maxwidth is the original width if it is less than linewidth
%% otherwise use linewidth (to make sure the graphics do not exceed the margin)
\makeatletter
\def\maxwidth{ %
  \ifdim\Gin@nat@width>\linewidth
    \linewidth
  \else
    \Gin@nat@width
  \fi
}
\makeatother

\definecolor{fgcolor}{rgb}{0.345, 0.345, 0.345}
\newcommand{\hlnum}[1]{\textcolor[rgb]{0.686,0.059,0.569}{#1}}%
\newcommand{\hlstr}[1]{\textcolor[rgb]{0.192,0.494,0.8}{#1}}%
\newcommand{\hlcom}[1]{\textcolor[rgb]{0.678,0.584,0.686}{\textit{#1}}}%
\newcommand{\hlopt}[1]{\textcolor[rgb]{0,0,0}{#1}}%
\newcommand{\hlstd}[1]{\textcolor[rgb]{0.345,0.345,0.345}{#1}}%
\newcommand{\hlkwa}[1]{\textcolor[rgb]{0.161,0.373,0.58}{\textbf{#1}}}%
\newcommand{\hlkwb}[1]{\textcolor[rgb]{0.69,0.353,0.396}{#1}}%
\newcommand{\hlkwc}[1]{\textcolor[rgb]{0.333,0.667,0.333}{#1}}%
\newcommand{\hlkwd}[1]{\textcolor[rgb]{0.737,0.353,0.396}{\textbf{#1}}}%

\usepackage{framed}
\makeatletter
\newenvironment{kframe}{%
 \def\at@end@of@kframe{}%
 \ifinner\ifhmode%
  \def\at@end@of@kframe{\end{minipage}}%
  \begin{minipage}{\columnwidth}%
 \fi\fi%
 \def\FrameCommand##1{\hskip\@totalleftmargin \hskip-\fboxsep
 \colorbox{shadecolor}{##1}\hskip-\fboxsep
     % There is no \\@totalrightmargin, so:
     \hskip-\linewidth \hskip-\@totalleftmargin \hskip\columnwidth}%
 \MakeFramed {\advance\hsize-\width
   \@totalleftmargin\z@ \linewidth\hsize
   \@setminipage}}%
 {\par\unskip\endMakeFramed%
 \at@end@of@kframe}
\makeatother

\definecolor{shadecolor}{rgb}{.97, .97, .97}
\definecolor{messagecolor}{rgb}{0, 0, 0}
\definecolor{warningcolor}{rgb}{1, 0, 1}
\definecolor{errorcolor}{rgb}{1, 0, 0}
\newenvironment{knitrout}{}{} % an empty environment to be redefined in TeX

\usepackage{alltt}
%\documentclass[a4paper,twoside,11pt,titlepage]{book}
\usepackage{listings}
\usepackage[utf8]{inputenc}
\usepackage[spanish]{babel}

% \usepackage[style=list, number=none]{glossary} %
%\usepackage{titlesec}
%\usepackage{pailatino}

\decimalpoint
\usepackage{dcolumn}
\newcolumntype{.}{D{.}{\esperiod}{-1}}
\makeatletter
\addto\shorthandsspanish{\let\esperiod\es@period@code}
\makeatother


%\usepackage[chapter]{algorithm}
\RequirePackage{verbatim}
%\RequirePackage[Glenn]{fncychap}
\usepackage{fancyhdr}
\usepackage{graphicx}
\usepackage{afterpage}

\usepackage{longtable}

\usepackage[pdfborder={000}]{hyperref} %referencia

% ********************************************************************
% Re-usable information
% ********************************************************************
\newcommand{\myTitle}{Título del proyecto\xspace}
\newcommand{\myDegree}{Grado en ...\xspace}
\newcommand{\myName}{Nombre Apllido1 Apellido2 (alumno)\xspace}
\newcommand{\myProf}{Nombre Apllido1 Apellido2 (tutor1)\xspace}
\newcommand{\myOtherProf}{Nombre Apllido1 Apellido2 (tutor2)\xspace}
%\newcommand{\mySupervisor}{Put name here\xspace}
\newcommand{\myFaculty}{Escuela Técnica Superior de Ingenierías Informática y de
Telecomunicación\xspace}
\newcommand{\myFacultyShort}{E.T.S. de Ingenierías Informática y de
Telecomunicación\xspace}
\newcommand{\myDepartment}{Departamento de ...\xspace}
\newcommand{\myUni}{\protect{Universidad de Granada}\xspace}
\newcommand{\myLocation}{Granada\xspace}
\newcommand{\myTime}{\today\xspace}
\newcommand{\myVersion}{Version 0.1\xspace}


\hypersetup{
pdfauthor = {\myName (email (en) ugr (punto) es)},
pdftitle = {\myTitle},
pdfsubject = {},
pdfkeywords = {palabra_clave1, palabra_clave2, palabra_clave3, ...},
pdfcreator = {LaTeX con el paquete ....},
pdfproducer = {pdflatex}
}

%\hyphenation{}


%\usepackage{doxygen/doxygen}
%\usepackage{pdfpages}
\usepackage{url}
\usepackage{colortbl,longtable}
\usepackage[stable]{footmisc}
%\usepackage{index}

%\makeindex
%\usepackage[style=long, cols=2,border=plain,toc=true,number=none]{glossary}
% \makeglossary

% Definición de comandos que me son tiles:
%\renewcommand{\indexname}{Índice alfabético}
%\renewcommand{\glossaryname}{Glosario}

\pagestyle{fancy}
\fancyhf{}
\fancyhead[LO]{\leftmark}
\fancyhead[RE]{\rightmark}
\fancyhead[RO,LE]{\textbf{\thepage}}
\renewcommand{\chaptermark}[1]{\markboth{\textbf{#1}}{}}
\renewcommand{\sectionmark}[1]{\markright{\textbf{\thesection. #1}}}

\setlength{\headheight}{1.5\headheight}

\newcommand{\HRule}{\rule{\linewidth}{0.5mm}}
%Definimos los tipos teorema, ejemplo y definición podremos usar estos tipos
%simplemente poniendo \begin{teorema} \end{teorema} ...
\newtheorem{teorema}{Teorema}[chapter]
\newtheorem{ejemplo}{Ejemplo}[chapter]
\newtheorem{definicion}{Definición}[chapter]

\definecolor{gray97}{gray}{.97}
\definecolor{gray75}{gray}{.75}
\definecolor{gray45}{gray}{.45}
\definecolor{gray30}{gray}{.94}

\lstset{ frame=Ltb,
     framerule=0.5pt,
     aboveskip=0.5cm,
     framextopmargin=3pt,
     framexbottommargin=3pt,
     framexleftmargin=0.1cm,
     framesep=0pt,
     rulesep=.4pt,
     backgroundcolor=\color{gray97},
     rulesepcolor=\color{black},
     %
     stringstyle=\ttfamily,
     showstringspaces = false,
     basicstyle=\scriptsize\ttfamily,
     commentstyle=\color{gray45},
     keywordstyle=\bfseries,
     %
     numbers=left,
     numbersep=6pt,
     numberstyle=\tiny,
     numberfirstline = false,
     breaklines=true,
   }

% minimizar fragmentado de listados
\lstnewenvironment{listing}[1][]
   {\lstset{#1}\pagebreak[0]}{\pagebreak[0]}

\lstdefinestyle{CodigoC}
   {
	basicstyle=\scriptsize,
	frame=single,
	language=C,
	numbers=left
   }
\lstdefinestyle{CodigoC++}
   {
	basicstyle=\small,
	frame=single,
	backgroundcolor=\color{gray30},
	language=C++,
	numbers=left
   }


\lstdefinestyle{Consola}
   {basicstyle=\scriptsize\bf\ttfamily,
    backgroundcolor=\color{gray30},
    frame=single,
    numbers=none
   }


\newcommand{\bigrule}{\titlerule[0.5mm]}


%Para conseguir que en las páginas en blanco no ponga cabecerass
\makeatletter
\def\clearpage{%
  \ifvmode
    \ifnum \@dbltopnum =\m@ne
      \ifdim \pagetotal <\topskip
        \hbox{}
      \fi
    \fi
  \fi
  \newpage
  \thispagestyle{empty}
  \write\m@ne{}
  \vbox{}
  \penalty -\@Mi
}
\makeatother

\usepackage{pdfpages}
\IfFileExists{upquote.sty}{\usepackage{upquote}}{}
\begin{document}
%\SweaveOpts{concordance=TRUE}
%\input{portada/portada}
%\chapter*{}
% \thispagestyle{empty}
%\cleardoublepage

%\thispagestyle{empty}

% \input{portada/portada_2}



% \cleardoublepage
\thispagestyle{empty}

\begin{center}
{\large\bfseries Estudio de optimización con incertidumbre en juegos de estrategia}\\
\end{center}
\begin{center}
Antonio Álvarez Caballero \\
\end{center}

%\vspace{0.7cm}
\noindent{\textbf{Palabras clave}: Estrategia, incertidumbre, aprendizaje sobre datos, predicción, clasificación}\\

\vspace{0.7cm}
\noindent{\textbf{Resumen}}\\

En este trabajo se realiza un estudio de la incertidumbre que presentan juegos
de estrategia en tiempo real como StarCraft, utilizando técnicas estadísticas
y de aprendizaje. Así se consigue evaluar la incertidumbre
que presentan este tipo de problemas para, en un paso posterior de optimización,
poder tenerla en cuenta y poder desarrollar agentes que sean capaces de evaluar
esta incertidumbre y actuar o modificar su conducta según lo necesite.

Se demostrará cómo el tiempo transcurrido es relevante en cuanto a la cantidad
de incertidumbre que presenta cada una de las partidas, aunque también se
expondrá que no es el factor más determinante: la información que un jugador
puede sacar del otro será incluso más importante.

\cleardoublepage


\thispagestyle{empty}


\begin{center}
{\large\bfseries Optimization study with uncertainty in strategy games}\\
\end{center}
\begin{center}
Antonio Álvarez Caballero \\
\end{center}

%\vspace{0.7cm}
\noindent{\textbf{Keywords}: Strategy, uncertainty, learning from data, prediction, classification}\\

\vspace{0.7cm}
\noindent{\textbf{Extended Abstract}}\\

\begin{otherlanguage}{british}

  In this project, the uncertainty that Real Time Strategy games like StarCraft present is studied, applying statistical and learning techniques . With all of this, uncertainty can be evaluated for next optimizations, and thus agents able to evaluate this uncertainty and modify its strategy could be developed.

  \subsubsection*{Statistical Learning}

  Firstly, the concept of \emph{statistical learning} is defined. It is a branch of Mathematics that aims to learn unknown functions from data using statistical techniques. This type of learning can be modelled mathematically to solve a large set of problems in Science, Engineering, Economy, etc.

  \subsubsection*{Feasibility of learning}
  Then, the feasibility of learning in finite hypothesis sets is proven, mainly using Hoeffding’s results. In-sample error and out-sample error concepts are introduced and used throughout  the entire project.

  \subsubsection*{Generalization: VC Theory}
  After that, generalizing to infinite hypothesis sets is needed. These are the most common sets in real life. For this purpose, the works of Vapnik and Chervonenkis are presented. Their work from the 60’s to the 90’s involved investigating in order to prove generalization and performing further relevant research for the statistical learning theory.
  Furthermore, the bias-var analysis is introduced in this chapter. It models the out-of-sample error using the bias and the variance of the model, considerably useful to develop new learning techniques and prevent overfitting.

  \subsubsection*{Overfitting}
  Overfitting is the main problem of learning. It occurs when in-sample error is great, but it does not generalizes at all. This is due to the fact that the model is too complex and able to fit the noise of the data, giving a final hypothesis that is worse than a simpler one.

  To combat overfitting Regularization could be used. Basically it constraints the algorithms to improve out-of-sample error instead of minimizing in-sample one. These methods are usually heuristics so there is a bit of art in this field.

  In this chapter, the validation concept and the cross-validation method for estimating out-of sample error are explained.


  \subsubsection*{Ensemble Methods}

  Lastly, in the mathematical background, the ensemble methods for learning are described. These methods combine weak learners (only slightly better than throwing a coin) to produce strong learners. Gradient Boosting is the reference in this project.

  \subsubsection*{Study of uncertainty}

  · Comprehension of the problem and data
  When anyone has to solve a data science problem the first step is understanding the data that could be obtained and the problem that will be solved. There are six StarCraft replays relational databases with a substantial amount of data, and thus the first step is understanding them and selecting a subset.

  \subsubsection*{Preprocessing}
  After that, the data have to be preprocessed in order to be able to work with statistical software. Dumping from a database, extracting the important features and organizing them are basically an art. Whoever handles the data best could have an important advantage.

  The next step is doing an Exploratory Data Analysis, since obtaining a general vision of the data and their distribution is essential. R is used for the analysis because it is a good reference in statistics software. Barplots and time plots are good tools to handle these data, and could give some ideas to fight them. As the dataset is too large, these ideas involve calculating the slope of the regression line for each feature and each player, and calculating the Area Under Curve of the feature graphs.

  \subsubsection*{Info extraction}
  Then, XGBoost, the library used in this project, is described. It is an open source library that implements the Gradient Boosting algorithm in a definitely efficient way. It is the state of the art in learning libraries and it has great importance in Kaggle competitions. It is used in some companies as well.

  It provides a very efficient and effective way to classify data with ensemble learning, with regularization techniques and parallel approaches; hence, it is rather fast and provides excellent results.

  \subsubsection*{Evaluation}
  Lastly, the results of playing with the data are explained, and conclusions will be presented in the last chapter.


\end{otherlanguage}


\chapter*{}
\thispagestyle{empty}

\noindent\rule[-1ex]{\textwidth}{2pt}\\[4.5ex]

Yo, \textbf{Antonio Álvarez Caballero}, alumno de la titulación Doble Grado en
Ingeniería Informática y Matemáticas de la \textbf{Escuela Técnica Superior
de Ingenierías Informática y de Telecomunicación} y de la \textbf{Facultad de Ciencias}
de la \textbf{Universidad de Granada}, con DNI XXXXXXXXX, autorizo la
ubicación de la siguiente copia de mi Trabajo Fin de Grado en la biblioteca de
ambos centros para que pueda ser consultada por las personas que lo deseen.

\vspace{6cm}

\noindent Fdo: Antonio Álvarez Caballero

\vspace{2cm}

\begin{flushright}
Granada a \_ de Julio de 2016.
\end{flushright}


\chapter*{}
\thispagestyle{empty}

\noindent\rule[-1ex]{\textwidth}{2pt}\\[4.5ex]

D. \textbf{Juan Julián Merelo Guervós}, Profesor del Departamento Arquitectura y Tecnología de Computadores de la Universidad de Granada.

\vspace{0.5cm}

\textbf{Informa:}

\vspace{0.5cm}

Que el presente trabajo, titulado \textit{\textbf{Estudio de optimización con incertidumbre en juegos de estrategia}},
ha sido realizado bajo su supervisión por \textbf{Antonio Álvarez Caballero}, y autoriza la defensa de dicho trabajo ante el tribunal
que corresponda.

\vspace{0.5cm}

Y para que conste, expide y firma el presente informe en Granada a \_ de Julio de 2016.

\vspace{1cm}

\textbf{El director:}

\vspace{5cm}

\noindent \textbf{Juan Julián Merelo Guervós}

\chapter*{Agradecimientos}
\thispagestyle{empty}

       \vspace{1cm}


Quería agradecer a la Universidad de Granada y a los profesores de ambas
titulaciones por darme la oportunidad de formarme con ellos. En especial, a mi
tutor JJ Merelo por la oportunidad de realizar este trabajo y su ayuda durante
el transcurso del proyecto.

También agradecer a mi familia, sin cuyo apoyo constante no podría haber llegado
hasta aquí.

A mis amigos, sin cuya ayuda no habría sobrevivido a más de una asignatura.

A mi pareja, sin cuyo infinito apoyo no habría conseguido llegar hasta el final.

%\frontmatter
\tableofcontents
%\listoffigures
%\listoftables
%
%\mainmatter
%\setlength{\parskip}{5pt}





\chapter{Introducción}


\chapter{Objetivos} 


\chapter{Desarrollo}
\section{Matemáticas}
Aquí Mates
\section{Ingeniería Informática}
Una vez se dispone de todas las herramientas matemáticas necesarias, se puede comenzar con el desarrollo de la parte de Ingeniería Informática que se aborda en el trabajo.

En primer lugar en cualquier problema de Ciencia de Datos, el primer paso es \emph{comprender el problema y los datos que se disponen}.

El problema a abordar es el siguiente: a partir de un conjunto de partidas antiguas de StarCraft \cite{dataset2014}, se busca predecir el momento en el que la partida está decidida con una confianza determinada. Para ello, partimos de 6 bases de datos relacionales (SQL) con gran cantidad de partidas almacenadas, cada una con muchas características a observar.

Cada una de estas bases de datos posee esta estructura:


\begin{figure}
    \centering
    \includegraphics[width=\maxwidth]{figure/Robertson14DatabaseDiagram}
    \caption{Base de datos de partidas de StarCraft}
\end{figure}



Una vez se tiene conocimiento del problema y un conjunto de datos, se debe decidir qué datos y características van a ser usados y de qué forma. El principal problema de este paso es conocer el conjunto de datos del que se dispone, ya que usualmente no es extraído por los investigadores.

Las características están sacadas casi en su totalidad directamente de valores que proporciona la API que permite interactuar con StarCraft, \emph{BWAPI}. Otros son datos derivados, como la distancia de un jugador a la base más cercana, por ejemplo.

Las características que van a ser usados en este trabajo son, principalmente, los recursos de cada jugador, sus batallones (que son medidos de una manera determinada que se explican con más detalle más adelante), sus construcciones, y los valores estimados de batallones y contrucciones que tienen un jugador del otro. Además, también se tiene en cuenta los recursos restantes del mapa que cada jugador estima que quedan.

Las características con las que se va a trabajar, según el diagrama mostrado anteriormente, son estas:

\begin{figure}
    \centering
    \includegraphics[width=\maxwidth]{figure/Robertson14DatabaseDiagramSeleccion}
    \caption{Base de datos de partidas de StarCraft con características seleccionadas}
\end{figure}


Estas características son, según cada tabla:

\begin{itemize}
  \item replay: Esta tabla contiene datos asociados a cada partida.
  \begin{itemize}
    \item ReplayID: Identificador de cada partida.
    \item Duration: Duración (en frames) de cada partida. 15 frames equivalen a 1 segundo.
  \end{itemize}
  \item playerreplay: Esta tabla contiene datos asociados a un jugador en una partida.
  \begin{itemize}
    \item PlayerReplayID: Identificador de un jugador en una partida.
    \item ReplayID: Identificador de partida asociado.
    \item Winner: Ganador de cada partida.
  \end{itemize}
  \item resourcechange: Esta tabla contiene datos asociados a cambios en los recursos de un jugador.
  \begin{itemize}
    \item PlayerReplayID: Identificador del jugador que produce un cambio.
    \item Frame: Frame en el que se produce un cambio.
    \item Minerals: Cantidad de minerales que tiene un jugador en ese momento.
    \item Gas: Cantidad de gas que tiene un jugador en ese momento.
    \item Supply: Capacidad de carga del jugador.
    \item TotalMinerals: Cantidad total de minerales que ha obtenido un jugador, sin contar gastos.
    \item TotalGas: Cantidad total de gas que ha obtenido un jugador, sin contar gastos.
    \item TotalSupply: Capacidad que ha obtenido un jugador, sin contar gastos.
  \end{itemize}
  \item regionvaluechange: Esta tabla contiene datos asociados a cambios de un jugador en una región del mapa determinada. Cada \emph{value}, que llamaremos de aquí en adelante \emph{valor}, es la suma del precio de una unidad en Minerales y Gas.
  \begin{itemize}
    \item PlayerReplayID: Identificador del jugador que produce un cambio.
    \item RegionID: Identificador de la región del mapa donde se produce un cambio.
    \item Frame: Frame en el que se produce el cambio.
    \item GroundUnitValue: Valor de las unidades terrestres en esta región.
    \item BuildingValue: Valor de las construcciones en esta región.
    \item AirUnitValue: Valor de las unidades aéreas en esta región.
    \item EnemyGroundUnitValue: Valor de las unidades terrestres del enemigo en esta región. Este valor es estimado, sólo se conoce lo que el jugador puede ver del enemigo.
    \item EnemyBuildingValue: Valor de las contrucciones del enemigo en esta región. Este valor es estimado, sólo se conoce lo que el jugador puede ver del enemigo.
    \item EnemyAirUnitValue: Valor de las unidades aéreas del enemigo en esta región. Este valor es estimado, sólo se conoce lo que el jugador puede ver del enemigo.
    \item ResourceValue: Valor de los recursos en esta región. Este valor es estimado, sólo se conoce lo que el jugador puede ver del mapa. Si el jugador no conoce una zona, estima que los recursos restantes es la totalidad de lo disponible en la región.
  \end{itemize}
\end{itemize}

Una vez se ha decidido qué vamos a usar, hay que pasar al cómo. Esta fase es el \emph{preprocesamiento de los datos}, que es la fase donde se organizan los datos, se corrigen si hubiera datos perdidos o ruido... para poder abordar el problema a resolver.

EoEo

Eodidi


\chapter{Conclusiones} 

%%\chapter{Conclusiones y Trabajos Futuros}
%
%
%%\nocite{*}
%\bibliography{bibliografia/bibliografia}\addcontentsline{toc}{chapter}{Bibliografía}
%\bibliographystyle{miunsrturl}
%
%\appendix
%\input{apendices/manual_usuario/manual_usuario}
%%\input{apendices/paper/paper}
%\input{glosario/entradas_glosario}
% \addcontentsline{toc}{chapter}{Glosario}
% \printglossary
\chapter*{}
\thispagestyle{empty}

\bibliography{bibliografia}{}
\bibliographystyle{plain}

\end{document}
